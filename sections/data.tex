\section{创建数据}
\subsection[DataFrame]{\inlinetext{DataFrame}}
数据帧 \inlinetext{DataFrame} 是二维数据结构, 即数据以行和列的表格方式排列. 数据帧 \inlinetext{DataFrame} 的功能特点:%
%
\begin{itemize}
    \item 潜在的列是不同的类型;
    \item 大小可变;
    \item 标记轴(行和列);
    \item 可以对行和列执行算术运算.
\end{itemize}

\subsubsection{创建}
Pandas 数据帧 \inlinetext{DataFrame} 可以使用各种输入创建, 如列表, 字典, 系列, \inlinetext{numpy.ndarray}, 数据帧 \inlinetext{DataFrame}.

创建基本数据帧是空数据帧, 如\cref{code:create empty data frame} 所示%
%
\begin{codebox}[
  label = code:create empty data frame,
  caption = 创建一个空的 \inlinetext{DataFrame}
]
>>> import pandas
>>> data_frame = pandas.DataFrame()
>>> print(data_frame)
Empty DataFrame
Columns: []
Index: []
\end{codebox}

可以使用单个列表或列表列表创建数据帧 \inlinetext{DataFrame}, 如\cref{code:create data frame from single list} 所示.%
%
\begin{codebox}[
  label = code:create data frame from single list,
  caption = 用单个列表创建 \inlinetext{DataFrame}
]
>>> import pandas
>>> data = [1,2,3,4,5]
>>> data_frame = pandas.DataFrame(data)
>>> print(data_frame)
   0
0  1
1  2
2  3
3  4
4  5
\end{codebox}

我们可以用 \inlinetext{column} 参数为列指定名称, 如\cref{code:create data frame with columns} 所示.%
%
\begin{codebox}[
  label = code:create data frame with columns,
  caption = 用 \inlinetext{columns} 参数为列指定名称
]
>>> import pandas
>>> data = [1, 2, 3, 4, 5] |\label{line:initialization}|
>>> data_frame = pandas.DataFrame(data, columns = ['number'])
>>> print(data_frame)
   number
0       1
1       2
2       3
3       4
4       5
\end{codebox}

其实\cref{code:create data frame with columns} 中代码的 \inlinepython{data = [1,2,3,4,5]} 是特例, 完整的写法如\cref{code:create data frame with columns'} 所示. 请注意\cref{code:create data frame with columns} 中的第 \ref{line:initialization} 行与\cref{code:create data frame with columns'} 中的第 \ref{line:initialization of data 2} 行的区别.%
%
\begin{codebox}[
  label = code:create data frame with columns',
  caption = 用 \inlinetext{columns} 参数为列指定名称,
]
>>> import pandas
>>> data = [[1], [2], [3], [4], [5]] |\label{line:initialization of data 2}|
>>> data_frame = pandas.DataFrame(data, columns = ['number'])
>>> print(data_frame)
   number
0       1
1       2
2       3
3       4
4       5
\end{codebox}

\subsection[Panel]{\inlinetext{Panel}}