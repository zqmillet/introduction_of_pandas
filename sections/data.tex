\section{创建数据}
\subsection[DataFrame]{\inlinetext{DataFrame}}
数据帧 \inlinetext{DataFrame} 是二维数据结构, 即数据以行和列的表格方式排列. 数据帧 \inlinetext{DataFrame} 的功能特点:%
%
\begin{itemize}
    \item 潜在的列是不同的类型;
    \item 大小可变;
    \item 标记轴(行和列);
    \item 可以对行和列执行算术运算.
\end{itemize}

\subsubsection{创建}
Pandas 数据帧 \inlinetext{DataFrame} 可以使用各种输入创建, 如列表, 字典, 系列, \inlinetext{numpy.ndarray}, 数据帧 \inlinetext{DataFrame}.

创建基本数据帧是空数据帧, 如\cref{code:create_empty_data_frame} 所示%
%
\begin{codebox}[
  label = code:create_empty_data_frame,
  caption = 创建一个空的 \inlinetext{DataFrame}
]
>>> import pandas
>>> data_frame = pandas.DataFrame()
>>> print(data_frame)
Empty DataFrame
Columns: []
Index: []
\end{codebox}

可以使用单个列表或列表列表创建数据帧 \inlinetext{DataFrame}, 如\cref{code:create_data_frame_from_single_list} 所示.%
%
\begin{codebox}[
  label = code:create_data_frame_from_single_list,
  caption = 用单个列表创建 \inlinetext{DataFrame}
]
>>> import pandas
>>> data = [1,2,3,4,5]
>>> data_frame = pandas.DataFrame(data)
>>> print(data_frame)
   0
0  1
1  2
2  3
3  4
4  5
\end{codebox}

我们可以用 \inlinetext{column} 参数为列指定名称, 如\cref{code:create_data_frame_with_columns_1} 所示.%
%
\begin{codebox}[
  label = code:create_data_frame_with_columns_1,
  caption = 用 \inlinetext{columns} 参数为列指定名称
]
>>> import pandas
>>> data = [1, 2, 3, 4, 5] |\label{line:initialization_of_data_1}|
>>> data_frame = pandas.DataFrame(data, columns = ['number'])
>>> print(data_frame)
   number
0       1
1       2
2       3
3       4
4       5
\end{codebox}%
%
其实\cref{code:create_data_frame_with_columns_1} 中代码的 \inlinepython{data = [1,2,3,4,5]} 是特例, 完整的写法如\cref{code:create_data_frame_with_columns_2} 所示. 请注意\cref{code:create_data_frame_with_columns_1} 中的第 \ref{line:initialization_of_data_1} 行与\cref{code:create_data_frame_with_columns_2} 中的第 \ref{line:initialization_of_data_2} 行的区别, 两种写法是等价的.%
%
\begin{codebox}[
  label = code:create_data_frame_with_columns_2,
  caption = 用 \inlinetext{columns} 参数为列指定名称,
]
>>> import pandas
>>> data = [[1], [2], [3], [4], [5]] |\label{line:initialization_of_data_2}|
>>> data_frame = pandas.DataFrame(data, columns = ['number'])
>>> print(data_frame)
   number
0       1
1       2
2       3
3       4
4       5
\end{codebox}

如果要产生一个两列的 \inlinetext{DataFrame}, 如\cref{code:create_data_frame_with_two_columns} 所示.%
%
\begin{codebox}[
  label = code:create_data_frame_with_two_columns,
  caption = 产生两列的 \inlinetext{DataFrame},
]
>>> import pandas
>>> data = [['Alex', 10], ['Bob', 12], ['Clarke', 13]]
>>> data_frame = pandas.DataFrame(data, columns = ['Name', 'Age'])
>>> print(data_frame)
     Name  Age
0    Alex   10
1     Bob   12
2  Clarke   13
\end{codebox}%
%
值得注意的是, \inlinetext{columns} 参数的尺寸与 \inlinetext{data} 中的元素尺寸一致. 我们可以利用 \inlinetext{index} 参数为每一行数据设置标签, 如\cref{code:create_data_frame_with_row_labels} 所示.%
%
\begin{codebox}[
  label = code:create_data_frame_with_row_labels,
  caption = 为 \inlinetext{DataFrame} 设置行标签,
]
>>> import pandas
>>> data = [['Alex', 10], ['Bob', 12], ['Clarke', 13]]
>>> data_frame = pandas.DataFrame(data, columns = ['Name', 'Age'], index = [1, 2, 3])
>>> print(data_frame)
     Name  Age
1    Alex   10
2     Bob   12
3  Clarke   13
\end{codebox}%
%
值得注意的是, \inlinetext{index} 参数的尺寸与 \inlinetext{data} 的尺寸一致.

除了 \inlinetext{list}, 我们也可以用 \inlinetext{dict} 创建 \inlinetext{DataFrame}, 如\cref{code:create_data_frame_from_dict} 所示.%
%
\begin{codebox}[
  label = code:create_data_frame_from_dict,
  caption = 利用 \inlinetext{dict} 创建 \inlinetext{DataFrame},
]
>>> import pandas
>>> data = {
...     'Name': ['Tom', 'Jack', 'Steve', 'Ricky'],
...     'Age':  [28,    34,     29,      42]
... }
>>> data_frame = pandas.DataFrame(data, index = ['rank1', 'rank2', 'rank3', 'rank4'])
>>> print(data_frame)
       Age   Name
rank1   28    Tom
rank2   34   Jack
rank3   29  Steve
rank4   42  Ricky
\end{codebox}

% \subsection[Panel]{\inlinetext{Panel}}
